\noindent \texttt{V: 24-03-2023 23:16:00}

\section*{Experimento: - Máquinas Eletrostáticas - Gerador Van De Graaff}

Universidade Federal de Santa Maria
Centro de Ciências Naturais e Exatas
Curso de Física 
FSC326 - Laboratório de Física III
Data de realização do Experimento: 23/03/2018
Professor: Hans Rogério Zimermann
Grupo de Pesquisa: (X) 1 ( ) 2 ( ) 3 ( ) 4
Matricula Turma Nome Completo
1 201610040 FSC1026 Guilherme Danezi Piccini
2
3
4
5
6
Roteiro: 01
Experimento: Máquinas Eletrostáticas
Requisitos Obrigatórios
Item Elemento Textual Nota
1 Capa Padrão: preenchimento completo e legível
2 Itens: organização e encadeamento lógico do trabalho.
3 Resumo: correspondência do resumo com o conteúdo do trabalho.
4 Introdução Teórica ao Tema: leis físicas do experimento abordadas e relacionadas com o 
experimento e clareza dos objetivos.
5 Procedimento experimental: descrição do procedimento utilizado incluindo relação do material 
utilizado, esquemas e figuras quando necessário.
6 Dados das medições: apresentação de todas as grandezas medidas e adotadas no 
experimento, com as respectivas unidades.
7 Análise dos dados e resultados: fórmulas e cálculos corretos, resultados apresentados com o 
uso adequado dos algarismos significativos e unidades de medidas.
8 Conclusões: discussão da validade ou não dos resultados encontrados, considerando-se a 
precisão dos equipamentos e valores de referências teóricas.
9 Bibliografia: é apresentada bibliografia pertinente.
Avaliação do Relatório: . 
Disciplina: FSC1026 – Física Geral e Experimental III
Esta atividade encontra-se em: http://portalfisica.com Disciplina FSC1026
Roteiro de Atividades - 01 (2018) 
Prof. Hans R Zimermann - hans@ufsm.br
26
ROTEIRO DE EXPERIMENTO - 01
Este roteiro tem objetivo de guiar as atividades de estudo (leitura, exercícios e experimentos). As atividades que serão usadas para avaliação 
serão disponibilizadas no http://www.portalfisica.com/fsc326.html e também no AVEA Moodle UFSM http://nte.ufsm.br/ na área da disciplina de 
Laboratório de Física III no decorrer do período dessas atividades.
A – MÁQUINAS ELETROSTÁTICAS – GERADOR DE VAN DE GRAAFF
1. Objetivo:
Verificar comportamento de cargas estáticas e as manifestações da Força Elétrica.
2. Fundamentos da eletrostática e do Gerador de Van De Graaff
Faça a revisão bibliográfica sobre o assunto
3. Experimento
MATERIAL:
- Cite os materiais e equipamentos utilizados
PROCEDIMENTO:
Descreva os procedimentos executados em aula pelo professor.
 Força Elétrica - Eletroscópio
 Poder das Pontas - “Para-raios”
 Vento Iônico
4. Relatório
a. Descreva o experimento em detalhe. Realçando os itens que mais lhe chamaram atenção.
Bibliografia
David Halliday, Robert Resnick, Jearl Walker; Fundamentos de Física, Volume 3, 8ª edição, São Paulo 2010.
Capítulo 1 - O Laboratório de Física (JURAITIS, K. R.; DOMICIANO, J. B.; Introdução ao 
Laboratório de Física Experimental, Londrina, PR, 2009)
Disciplina: FSC1026 – Física Geral e Experimental III
Esta atividade encontra-se em: http://portalfisica.com Disciplina FSC1026
Roteiro de Atividades - 01 (2018) 
Prof. Hans R Zimermann - hans@ufsm.br
36
MÁQUINAS ELETROSTÁTICAS – GERADOR DE VAN DE GRAAFF
1. INTRODUÇÃO E OBJETIVO
Por volta de 1930, o engenheiro estado-unidense Robert Jeminson Van de Graaff inventou uma 
máquina capaz de demonstrar de forma visível a ação da eletricidade a partir da transferência das 
cargas elétricas de um corpo eletrizado para outro. Em sua homenagem, esse aparelho foi batizado
de Gerador de Van de Graaff. Seus princípios ainda são utilizados atualmente, a máquina atua na 
física nuclear, em versões mais potentes, para produzir tensões muito elevadas em aceleradores de 
partículas, assim como na medicina e na indústria de alta tecnologia.
Essa máquina também foi fundamental para atingir o objetivo do experimento realizado em 
laboratório. Sob tutoria do Professor Hans Rogério Zimermann, a turma de Física Experimental III –
Eng Civil, utilizou o Gerador de Van de Graaff para executar três procedimentos em laboratório e 
verificar as manifestações da força elétrica e o comportamento das cargas estáticas submetidas à 
situações de transferência de cargas entre dois corpos, bem como para analisar a disposição dessas 
cargas de acordo com o formato de cada corpo e seus efeitos resultantes.
2. FUNDAMENTOS DA ELETROSTÁTICA E GERADOR DE VAN DE GRAAFF
Os fundamentos básicos da eletrostática regem o funcionamento do Gerador de Van de Graaff, 
sendo eles: o Princípio da atração e repulsão, responsável por demonstrar que cargas de mesmo 
sinal tendem a se repelir e cargas de sinais contrários tendem a se atrair. O Princípio da conservação 
de cargas, o qual define que a quantidade total de cargas de um sistema é sempre constante. E os 
tipos de eletrização (Atrito, Contato e Indução) que definem como acontece a transferência das 
cargas.
O gerador de Van de Graaff funciona através da movimentação de uma correia que é eletrizada por 
atrito na parte inferior do aparelho. Ao atingir a parte superior, as cargas elétricas que surgiram com 
o processo de eletrização por Atrito, são transferidas para a superfície interna do metal, sendo então 
distribuídas para toda a superfície da esfera metálica, ficando carregada de cargas elétricas. Se 
durante o funcionamento do gerador aproximarmos o dedo ou um objeto de metal perceberemos 
leves descargas elétricas que ocorrem em razão da diferença de potencial. 
Esse gerador é composto por:
 Um motor; 
 Dois cilindros; 
 Um conjunto de correias; 
 Um conjunto de escovas; 
 Um terminal de saída, que na maioria das vezes é uma grande esfera de metal ou de alumínio.
Disciplina: FSC1026 – Física Geral e Experimental III
Esta atividade encontra-se em: http://portalfisica.com Disciplina FSC1026
Roteiro de Atividades - 01 (2018) 
Prof. Hans R Zimermann - hans@ufsm.br
46
3. EXPERIMENTO
Realizou-se três experimentos em laboratório para observar o comportamento das cargas elétricas, 
sendo eles:
 Força Elétrica – Eletroscópio
 Poder das Pontas - “Para-raios”
 Vento Iônico
3.1 Força elétrica – Eletroscópio:
FUNÇÃO:
Esse experimento teve a função de verificar se um corpo está ou não eletrizado, assim como 
observar a intensidade da sua eletrização
MATERIAIS UTILIZADOS: 
 Gerador de Van de Graaff
 Material isolante
 Fios condutores
 Eletroscópio
 Corpo condutor que será eletrizado
PROCEDIMENTO: 
Ligou-se o Gerador de Van de Graaff na corrente elétrica, a qual fez a correia movimentar-se entre 
as escovas, eletrizando-a por atrito. As cargas negativas chegaram até a esfera de alumínio por 
contato.
Com o gerador eletrizado, aproximou-se uma outra esfera condutora, a qual teve suas cargas 
separadas por Indução. Após isso, ligou-se a parte positiva da esfera na terra para descarregar as 
cargas e o corpo também ficar eletrizado negativamente
Sob essas condições, foi possível observar a repulsão das cargas nas extremidades do
eletroscópio, quando aproximado de algum dos corpos. Informando-nos que o corpo estava 
eletrizado.
3.2 Poder das pontas – “Para-raios”
FUNÇÃO:
Disciplina: FSC1026 – Física Geral e Experimental III
Esta atividade encontra-se em: http://portalfisica.com Disciplina FSC1026
Roteiro de Atividades - 01 (2018) 
Prof. Hans R Zimermann - hans@ufsm.br
56
Esse experimento teve a função de verificar a distribuição das cargas elétricas em um corpo que 
possui extremidades pontiagudas
MATERIAIS UTILIZADOS: 
 Gerador de Van de Graaff
 Fios condutores
 “Tachinha” pontiaguda
PROCEDIMENTO:
Antes de ligar o Gerador de Van de Graaff na corrente elétrica, posicionou-se a “tachinha”
pontiaguda no topo da esfera metálica.
Assim que o Gerador começou a funcionar, percebeu-se que não houve acúmulo considerável de 
cargas na superfície da esfera oca, em comparação com os procedimentos anteriores.
Em contrapartida, havia grandes concentrações de carga ao redor da “tachinha” pontiaguda. Isso 
deve-se ao princípio do “Poder das Pontas”, que define que as cargas elétricas de um corpo se
concentram nas regiões mais pontiagudas, fazendo com que o campo elétrico nas vizinhanças 
dessas pontas atinja determinado valor, ionizando o ar em sua volta, tornando-o condutor.
Esse princípio é utilizado nos para-raios, fazendo com que a nuvem eletrizada descarregue suas 
cargas nas pontas do para-raio. Como o para-raio está ligado a terra, as cargas elétricas recebidas 
são transferidas ao solo sem nenhum problema. 
3.3 Vento Iônico
FUNÇÃO:
Esse experimento teve a função de observar a repulsão das cargas elétricas gerada pela ionização 
do ar.
MATERIAIS UTILIZADOS: 
 Gerador de Van de Graaff
 Fios condutores
 “Tachinha”
 Torniquete com 4 pontas em forma de “Z”
PROCEDIMENTO: 
Antes de ligar o Gerador de Van de Graaff na corrente elétrica, posicionou-se a “tachinha” no topo 
Disciplina: FSC1026 – Física Geral e Experimental III
Esta atividade encontra-se em: http://portalfisica.com Disciplina FSC1026
Roteiro de Atividades - 01 (2018) 
Prof. Hans R Zimermann - hans@ufsm.br
66
da esfera metálica com o torniquete conectado à ponta da tachinha. 
Quando o gerador foi ligado na corrente elétrica, observou-se que o torniquete começou a girar. 
Esse efeito deve-se à ionização do ar nas pontas do torniquete, a qual concentrou as cargas 
devido ao “Poder das pontas” e gerou repulsão dos íons de mesmo sinal, determinando a rotação 
acelerada nas pontas. 