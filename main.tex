\documentclass[a4paper,10pt]{article}
\usepackage{geometry}
 \geometry{ a4paper, total={170mm,257mm}, left=20mm, top=20mm}
\usepackage{verbatim}
\usepackage[utf8]{inputenc}
\usepackage{amsmath}
%\usepackage{gensymb}
\usepackage{verbatim} % env for block comment 
\usepackage{ragged2e} % para usar flusleft
\usepackage[brazilian]{babel}
\usepackage{hyperref}
\usepackage{array}
\usepackage{multirow}
\usepackage{tabu}
\usepackage{booktabs}
\usepackage{parskip}
\hypersetup{
    colorlinks=true,
    linkcolor=red,
    filecolor=magenta,      
    urlcolor=blue,
    pdftitle={zrhans@gmail.com},
    %bookmarks=true,
    pdfpagemode=FullScreen,}
    
\urlstyle{same}
\usepackage{minted}     % Realce de código
\usemintedstyle{tango}  % Tema do realce de código

\usepackage{wrapfig}
\usepackage[export]{adjustbox} % Ajuste das imagens To change the default alignment of a image from left or right 

\usepackage{tikz,tkz-euclide}
\usepackage{circuitikz} 


\usepackage{fancyhdr}
 
\pagestyle{fancy}
\fancyhf{}
\renewcommand{\headrulewidth}{2pt}
\renewcommand{\footrulewidth}{1pt}
\rhead{\href{http://portalfisica.com}{portalfisica.com}}
\lhead{UFSM}
\lfoot{Prof. Hans R. Zimermann - hans@ufsm.br}
\rfoot{Página - \thepage}

\begin{document}

\begin{wrapfigure}{l}{0.2\textwidth}
\includegraphics[width=.7\linewidth]{img/atm} 
\end{wrapfigure}

\noindent
\textsc{UFSM00029 - Física Experimental III \\ FSC1026 - Física Geral Experimental III \\ FSC326  \, -  Laboratório de Física III} \\
Engenharias e Física \\
Semestre - 1.2023 \\
Prof. Hans R. Zimermann \\

\noindent \rule{\linewidth}{0.5mm}

\begin{flushright}
{\color{orange}\textit{\mbox{"A perfeição é atingida não quando não se tem mais o que colocar,} \\ \mbox{ mas sim quando não se tem mais o que tirar." \;- Antoine de Saint-Exupéry}} }
\end{flushright}

%%%%%%%%%%%%%%%%%%%%%%%%%%%%%%%%%%%%%%%
%                PARTES
%%%%%%%%%%%%%%%%%%%%%%%%%%%%%%%%%%%%%%%

\noindent \texttt{V: 24-03-2023 23:16:00}

\section*{Experimento: - Máquinas Eletrostáticas - Gerador Van De Graaff}

Universidade Federal de Santa Maria
Centro de Ciências Naturais e Exatas
Curso de Física 
FSC326 - Laboratório de Física III
Data de realização do Experimento: 23/03/2018
Professor: Hans Rogério Zimermann
Grupo de Pesquisa: (X) 1 ( ) 2 ( ) 3 ( ) 4
Matricula Turma Nome Completo
1 201610040 FSC1026 Guilherme Danezi Piccini
2
3
4
5
6
Roteiro: 01
Experimento: Máquinas Eletrostáticas
Requisitos Obrigatórios
Item Elemento Textual Nota
1 Capa Padrão: preenchimento completo e legível
2 Itens: organização e encadeamento lógico do trabalho.
3 Resumo: correspondência do resumo com o conteúdo do trabalho.
4 Introdução Teórica ao Tema: leis físicas do experimento abordadas e relacionadas com o 
experimento e clareza dos objetivos.
5 Procedimento experimental: descrição do procedimento utilizado incluindo relação do material 
utilizado, esquemas e figuras quando necessário.
6 Dados das medições: apresentação de todas as grandezas medidas e adotadas no 
experimento, com as respectivas unidades.
7 Análise dos dados e resultados: fórmulas e cálculos corretos, resultados apresentados com o 
uso adequado dos algarismos significativos e unidades de medidas.
8 Conclusões: discussão da validade ou não dos resultados encontrados, considerando-se a 
precisão dos equipamentos e valores de referências teóricas.
9 Bibliografia: é apresentada bibliografia pertinente.
Avaliação do Relatório: . 
Disciplina: FSC1026 – Física Geral e Experimental III
Esta atividade encontra-se em: http://portalfisica.com Disciplina FSC1026
Roteiro de Atividades - 01 (2018) 
Prof. Hans R Zimermann - hans@ufsm.br
26
ROTEIRO DE EXPERIMENTO - 01
Este roteiro tem objetivo de guiar as atividades de estudo (leitura, exercícios e experimentos). As atividades que serão usadas para avaliação 
serão disponibilizadas no http://www.portalfisica.com/fsc326.html e também no AVEA Moodle UFSM http://nte.ufsm.br/ na área da disciplina de 
Laboratório de Física III no decorrer do período dessas atividades.
A – MÁQUINAS ELETROSTÁTICAS – GERADOR DE VAN DE GRAAFF
1. Objetivo:
Verificar comportamento de cargas estáticas e as manifestações da Força Elétrica.
2. Fundamentos da eletrostática e do Gerador de Van De Graaff
Faça a revisão bibliográfica sobre o assunto
3. Experimento
MATERIAL:
- Cite os materiais e equipamentos utilizados
PROCEDIMENTO:
Descreva os procedimentos executados em aula pelo professor.
 Força Elétrica - Eletroscópio
 Poder das Pontas - “Para-raios”
 Vento Iônico
4. Relatório
a. Descreva o experimento em detalhe. Realçando os itens que mais lhe chamaram atenção.
Bibliografia
David Halliday, Robert Resnick, Jearl Walker; Fundamentos de Física, Volume 3, 8ª edição, São Paulo 2010.
Capítulo 1 - O Laboratório de Física (JURAITIS, K. R.; DOMICIANO, J. B.; Introdução ao 
Laboratório de Física Experimental, Londrina, PR, 2009)
Disciplina: FSC1026 – Física Geral e Experimental III
Esta atividade encontra-se em: http://portalfisica.com Disciplina FSC1026
Roteiro de Atividades - 01 (2018) 
Prof. Hans R Zimermann - hans@ufsm.br
36
MÁQUINAS ELETROSTÁTICAS – GERADOR DE VAN DE GRAAFF
1. INTRODUÇÃO E OBJETIVO
Por volta de 1930, o engenheiro estado-unidense Robert Jeminson Van de Graaff inventou uma 
máquina capaz de demonstrar de forma visível a ação da eletricidade a partir da transferência das 
cargas elétricas de um corpo eletrizado para outro. Em sua homenagem, esse aparelho foi batizado
de Gerador de Van de Graaff. Seus princípios ainda são utilizados atualmente, a máquina atua na 
física nuclear, em versões mais potentes, para produzir tensões muito elevadas em aceleradores de 
partículas, assim como na medicina e na indústria de alta tecnologia.
Essa máquina também foi fundamental para atingir o objetivo do experimento realizado em 
laboratório. Sob tutoria do Professor Hans Rogério Zimermann, a turma de Física Experimental III –
Eng Civil, utilizou o Gerador de Van de Graaff para executar três procedimentos em laboratório e 
verificar as manifestações da força elétrica e o comportamento das cargas estáticas submetidas à 
situações de transferência de cargas entre dois corpos, bem como para analisar a disposição dessas 
cargas de acordo com o formato de cada corpo e seus efeitos resultantes.
2. FUNDAMENTOS DA ELETROSTÁTICA E GERADOR DE VAN DE GRAAFF
Os fundamentos básicos da eletrostática regem o funcionamento do Gerador de Van de Graaff, 
sendo eles: o Princípio da atração e repulsão, responsável por demonstrar que cargas de mesmo 
sinal tendem a se repelir e cargas de sinais contrários tendem a se atrair. O Princípio da conservação 
de cargas, o qual define que a quantidade total de cargas de um sistema é sempre constante. E os 
tipos de eletrização (Atrito, Contato e Indução) que definem como acontece a transferência das 
cargas.
O gerador de Van de Graaff funciona através da movimentação de uma correia que é eletrizada por 
atrito na parte inferior do aparelho. Ao atingir a parte superior, as cargas elétricas que surgiram com 
o processo de eletrização por Atrito, são transferidas para a superfície interna do metal, sendo então 
distribuídas para toda a superfície da esfera metálica, ficando carregada de cargas elétricas. Se 
durante o funcionamento do gerador aproximarmos o dedo ou um objeto de metal perceberemos 
leves descargas elétricas que ocorrem em razão da diferença de potencial. 
Esse gerador é composto por:
 Um motor; 
 Dois cilindros; 
 Um conjunto de correias; 
 Um conjunto de escovas; 
 Um terminal de saída, que na maioria das vezes é uma grande esfera de metal ou de alumínio.
Disciplina: FSC1026 – Física Geral e Experimental III
Esta atividade encontra-se em: http://portalfisica.com Disciplina FSC1026
Roteiro de Atividades - 01 (2018) 
Prof. Hans R Zimermann - hans@ufsm.br
46
3. EXPERIMENTO
Realizou-se três experimentos em laboratório para observar o comportamento das cargas elétricas, 
sendo eles:
 Força Elétrica – Eletroscópio
 Poder das Pontas - “Para-raios”
 Vento Iônico
3.1 Força elétrica – Eletroscópio:
FUNÇÃO:
Esse experimento teve a função de verificar se um corpo está ou não eletrizado, assim como 
observar a intensidade da sua eletrização
MATERIAIS UTILIZADOS: 
 Gerador de Van de Graaff
 Material isolante
 Fios condutores
 Eletroscópio
 Corpo condutor que será eletrizado
PROCEDIMENTO: 
Ligou-se o Gerador de Van de Graaff na corrente elétrica, a qual fez a correia movimentar-se entre 
as escovas, eletrizando-a por atrito. As cargas negativas chegaram até a esfera de alumínio por 
contato.
Com o gerador eletrizado, aproximou-se uma outra esfera condutora, a qual teve suas cargas 
separadas por Indução. Após isso, ligou-se a parte positiva da esfera na terra para descarregar as 
cargas e o corpo também ficar eletrizado negativamente
Sob essas condições, foi possível observar a repulsão das cargas nas extremidades do
eletroscópio, quando aproximado de algum dos corpos. Informando-nos que o corpo estava 
eletrizado.
3.2 Poder das pontas – “Para-raios”
FUNÇÃO:
Disciplina: FSC1026 – Física Geral e Experimental III
Esta atividade encontra-se em: http://portalfisica.com Disciplina FSC1026
Roteiro de Atividades - 01 (2018) 
Prof. Hans R Zimermann - hans@ufsm.br
56
Esse experimento teve a função de verificar a distribuição das cargas elétricas em um corpo que 
possui extremidades pontiagudas
MATERIAIS UTILIZADOS: 
 Gerador de Van de Graaff
 Fios condutores
 “Tachinha” pontiaguda
PROCEDIMENTO:
Antes de ligar o Gerador de Van de Graaff na corrente elétrica, posicionou-se a “tachinha”
pontiaguda no topo da esfera metálica.
Assim que o Gerador começou a funcionar, percebeu-se que não houve acúmulo considerável de 
cargas na superfície da esfera oca, em comparação com os procedimentos anteriores.
Em contrapartida, havia grandes concentrações de carga ao redor da “tachinha” pontiaguda. Isso 
deve-se ao princípio do “Poder das Pontas”, que define que as cargas elétricas de um corpo se
concentram nas regiões mais pontiagudas, fazendo com que o campo elétrico nas vizinhanças 
dessas pontas atinja determinado valor, ionizando o ar em sua volta, tornando-o condutor.
Esse princípio é utilizado nos para-raios, fazendo com que a nuvem eletrizada descarregue suas 
cargas nas pontas do para-raio. Como o para-raio está ligado a terra, as cargas elétricas recebidas 
são transferidas ao solo sem nenhum problema. 
3.3 Vento Iônico
FUNÇÃO:
Esse experimento teve a função de observar a repulsão das cargas elétricas gerada pela ionização 
do ar.
MATERIAIS UTILIZADOS: 
 Gerador de Van de Graaff
 Fios condutores
 “Tachinha”
 Torniquete com 4 pontas em forma de “Z”
PROCEDIMENTO: 
Antes de ligar o Gerador de Van de Graaff na corrente elétrica, posicionou-se a “tachinha” no topo 
Disciplina: FSC1026 – Física Geral e Experimental III
Esta atividade encontra-se em: http://portalfisica.com Disciplina FSC1026
Roteiro de Atividades - 01 (2018) 
Prof. Hans R Zimermann - hans@ufsm.br
66
da esfera metálica com o torniquete conectado à ponta da tachinha. 
Quando o gerador foi ligado na corrente elétrica, observou-se que o torniquete começou a girar. 
Esse efeito deve-se à ionização do ar nas pontas do torniquete, a qual concentrou as cargas 
devido ao “Poder das pontas” e gerou repulsão dos íons de mesmo sinal, determinando a rotação 
acelerada nas pontas. 

%\noindent \texttt{V: 25-10-2022 17:28:00}
\section*{Experimento: - CAPACITÂNCIA}
	\section{Objetivos}
	
	Estudar o efeito capacitivo entre as placas de um capacitor variável de placas paralelas - Figura \ref{fig:cidepe-capacitor}, verificando a relação da capacitância $C=\epsilon_{0} A/d$ em função da distância $d$ entre as placas; Medir a constante de permissividade ${\epsilon}_{0}$ e medir a constante dielétrica de distintos materiais isolantes como papel, \href{https://pt.wikipedia.org/wiki/Espuma_vin\%C3\%ADlica_acetinada}{ E.V.A (\textit{Ethylene Vinyl Acetate})} e isopor.	
	\section{Material utilizado}
	
\begin{wrapfigure}{r}{0.7\textwidth}
    \centering
    \includegraphics[width=.36\textwidth]{img/cidepe-capacitor.jpg}
    \caption{Capacitor de placas paralelas com separação variável montado em isolante acrílico. \textit{Fonte: Cidepe}.}
    \label{fig:cidepe-capacitor}
\end{wrapfigure}
	
	\begin{itemize}
		\item[a)] Um capacitor variável de placas paralelas: 2,3 pF - 280 pF;
		\item[b)] Um medidor digital de capacitância;
		\item[c)] Dois fios ou cabos condutores;
		\item[d)] Um papel milimetrado;
		\item[e)] Folhas de papel ou papelão;
		\item[f)] Lâminas de isolantes E.V.A, EPS (Isopor), Vidro, Acrílico,etc;
		\item[j)] Uma régua ou trena ou paquímetro.
		
	\end{itemize}
	
\begin{comment}
\begin{figure}[H]
	\centering
	\includegraphics[scale=.3]{img/cidepe-capacitor.jpg}
	\caption{Capacitor de placas paralelas com separação variável montado em isolante acrílico. Fonte: Cidepe.}
	\label{fig:imagem1}
\end{figure}
\end{comment}


\section{Fundamentos teóricos}
	
	Se preenchermos o espaço entre as placas de um capacitor com um dielétrico (isolante), o que acontecerá com a capacitância? \textbf{Michel Faraday} (1791-1867) investigou este assunto pela primeira vez em 1837. Usando equipamento simples, ele descobriu que a capacitância aumentava por um fator $\kappa$, que ele chamou de constante dielétrica do material isolante. Outro efeito importante na introdução do dielétrico no capacitor é limitar a diferença de potencial que se pode aplicar entre as placas a um certo valor ${V}_{max}$, chamado de potencial de ruptura. Se este valor for excedido, o material dielétrico se romperá e formará um caminho condutor ({\color{red}\textbf{arco elétrico}}) entre as placas. Logo, todo material dielétrico possui uma rigidez dielétrica característica, que é o valor máximo do campo elétrico que um isolante pode tolerar sem se romper e se tornar um condutor (para o ar a rigidez dielétrica é $~3\times10^{6} \,V.m^{-1}$, ou seja tensões ou diferenças de potencial ${V}_{max}$ acima desses valores, permitem a ocorrência de um \textbf{arco elétrico} ou popularmente \textit{raio/faísca/centelha}). Na Tabela \ref{tab:rigidez-materiais} são mostradas algumas constantes dielétricas importantes.
	
	%tabela 1
	\begin{table}[H]
		\centering
	\begin{tabular}{l|c|c}
		\hline 
		Material & Constante Dielétrica, $\kappa$ & Rigidez Dielétrica $E_{max} (10^{6}\,V.m^{-1})$ \\ 
		\hline 
		Ar(1 atm) & 1,0006 & 3 \\ 
        Madeira & & 10 \\
        Borracha & & 12 \\
	Papel & 3,5 & 16 \\ 
        Poliéster & 3,6 & 21,7 \\ 
        Vidro & & 30 \\
        Mica & & 60 \\
        Teflon & & 80 \\
        
		\hline 
	\end{tabular} 
		\caption{Constante dielétrica e rigidez dielétrica}
		\label{tab:rigidez-materiais}
	\end{table}

	\section{Procedimento experimental}
	
	\subsection{A - MEDIDA DA PERMISSIVIDADE ELÉTRICA ($\epsilon_{0}$)}
	
	\begin{itemize}
		\item[a)] Certifique-se de que o capacitor esteja descarregado, fazendo contato entre as duas placas por meio de um fio ou cabo condutor;
		\item[b)] Meça o diâmetro, calcule o raio e com este a área das placas do capacitor (use: $A=\pi r^2$, onde $r$ é o \textbf{raio} das placas). Anote os resultados na Tabela \ref{tab:area-placas};
		
		%tabela 2
		\begin{table}[H]
			\centering
		\begin{tabular}{|c|c|}
			\hline 
			Diâmetro das Placas ($m$) & Área das Placas do Capacitor ($m^2$)\\ 
			\hline
			&  \\ 
			\hline 
		\end{tabular}
			\caption{\label{tab:area-placas} Diâmetro e área do capacitor}
		\end{table}
		
		\item[c)] Fazer a conexão do medidor de capacitância nas placas do capacitor. \textbf{Zerar o aparelho antes de fazer a medida};
		\item[d)] Com a chave seletora do medidor em 200 pF, estabeleça um espaçamento aproximado de 1,0 ou 10 mm entre as placas do capacitor. Anote o valor da capacitância $C_{exp}$ na Tabela \ref{tab:cte-dieletrica}; 
		\item[e)] Calcule a constante de permissividade (use a relação: $ \epsilon_{0}= C_{exp} \, d / A. \kappa_{ar} $). Anote o resultado na Tabela \ref{tab:cte-dieletrica};
		\item[f)] Calcule o erro experimental, entre o valor teórico (da literatura), e o valor experimental (medido). Anote o resultado na Tabela \ref{tab:cte-dieletrica}. 
	
		%tabela 3
		\begin{table}[H]
			\centering
		\begin{tabular}{|c|c|c|c|c|}
			\hline 
			$C_{exp}$ (pF) & $\epsilon_{0}$ (Teórico)(pF) & $\epsilon_{0}$ (experimental)(pF) & Erro (Absoluto) & Erro (\%)  \\ 
			\hline 
			& 8,85 & & & \\ 
			\hline 
		\end{tabular}
			\caption{Medição da constante dielétrica}
			\label{tab:cte-dieletrica}
		\end{table} 
	\end{itemize}
	
	\subsection{B - VARIAÇÃO DA CAPACITÂNCIA COM A SEPARAÇÃO ENTRE AS PLACAS}
	
	\begin{itemize}
		\item[a)] Certifique-se de que o capacitor esteja descarregado;
		\item[b)] Com a chave seletora do medidor em 200 pF, varie a distância entre as placas de 1 mm em 1 mm até 10 mm (5 mm em 5 mm até 50 mm no Capacitor \href{https://www.cidepe.com.br/index.php/br/produtos-interna/capacitor-variavel-de-placas-paralelas-e-cabos-0-a-255-pf-1875}{Cidepe EQ065D} ou similar). Para cada variação meça a capacitância correspondente ({\color{red}\textbf{Para uma melhor precisão, a partir da segunda medida selecione a posição da chave em $200 \,\mu F$}}). Anote os valores das capacitâncias $ C_{exp} $ na Tabela \ref{tab:c-versus-inv-d};

		%tabela 4
		\begin{table}[H]
			\centering
		\begin{tabu}{| l | X[c] | X[c] | X[c] | X[c] | X[c] | X[c] | X[c] | X[c] | X[c] | X[c] |}
			\hline 
			d ($mm$) &  &  &  &  &  &  &  &  &  &  \\ 
			\hline 
			$C_{exp} \,(pF$) &  &  &  &  &  &  &  &  &  &  \\ 
			\hline 
			1/d ($mm^{-1}$) &  &  &  &  &  &  &  &  &  &  \\ 
			\hline 
		\end{tabu} 
			\caption{Variação da capacitância em função de d e 1/d}
			\label{tab:c-versus-inv-d}
		\end{table} 
	
		\item[c)] Faça um gráfico, digital ou em papel milimetrado, de $ C_{exp} \times d $ e em seguida de $ C_{exp} \times (1/d) $;
		\item[d)] Obtenha o coeficiente angular \textbf{$\alpha$} da reta $ C_{exp} \times (1/d) $ e compare com o valor do produto $ \kappa_{ar} \epsilon_{0} A$. Anote os valores na Tabela \ref{tab:coeficientes}; 
		\item[e)] Calcule o erro experimental absoluto e percentual.
		
		%tabela 5
		\begin{table}[H]
			\centering
		\begin{tabular}{|c|c|c|c|}
			\hline 
			Coeficiente angular $\alpha$ (pF.m) & $\kappa_{ar} \epsilon_{0} A$ (Teórico)(pFm) & Erro (Absoluto) & Erro (\%) \\ 
			\hline 
			&  &  & \\ 
			\hline 
		\end{tabular}
			\caption{Coeficientes e constantes}
			\label{tab:coeficientes}
		\end{table} 
		
	\end{itemize}
	
	\subsection{C - MEDIDAS DAS CONSTANTES DIELÉTRICAS DOS ISOLANTES}
	
	\begin{itemize}
		\item[a)] Certifique-se de que o capacitor esteja descarregado;
		\item[b)] Escolher uma placa/isolante dielétrico, folha de papel. Em seguida, medir sua espessura com um paquímetro (ou um micrômetro);
		\item[c)] Meça a capacitância do ar $C_{ar}$, para uma separação miníma entre as placas, aproximadamente ~1 mm (ou ~10mm no \href{https://www.cidepe.com.br/index.php/br/produtos-interna/capacitor-variavel-de-placas-paralelas-e-cabos-0-a-255-pf-1875}{Cidepe EQ065D} ou similar) com o medidor ({\color{red}\textbf{selecione uma escala de 200 pF}});
		\item[d)] Insira o dielétrico entre as placas do capacitor, em uma posição firme, e meça a capacitância equivalente, ${C}_{d+ar}$ com o medidor ({\color{red}\textbf{selecione uma escala de $200 \,\mu F$}}). \hbox{\textbf{Obs: $C_{d}$ = capacitância com dielétrico}};
		
		\item[e)] Calcule a capacitância da placa dielétrica usando a relação: $ 1/C_{placa}=1/C_{d+ar}-1/C_{ar}$. Anote o valor na Tabela \ref{tab:cte-dieletrica-isolantes};
		\item[f)] Calcule a constante dielétrica do isolante, $ \kappa_{isolante}=C_{d}.d{/\epsilon_{0}A}$;
		\item[g)] Calcule erro experimental entre as constantes teórica (literatura) e experimental;
			\item[h)] Repita os passos a-g para os outros isolantes.
		
		%tabela 6
		\begin{table}[H]
		\centering
		\begin{tabular}{|c|c|c|c|c|c|c|c|}
			\hline 
			$d$ (mm) & $C_{ar}$ (pF) & $C_{d+ar}$ (pF) & $\kappa_{isolante}$ (teórico) &  $\kappa_{isolante}$ (exp.) & \|Erro\|  & Erro (\%) \\ 
			\hline 
			&  &  &  &  &  &\\ 
			\hline 
		\end{tabular} 
			\caption{Constante dielétrica dos isolantes}
			\label{tab:cte-dieletrica-isolantes}
		\end{table}
	\end{itemize}

\section{Exercícios}

\begin{itemize}
	\item[a)] Justifique os erros observados no experimento;
	
	\item[b)] Qual o valor da constante de permissividade elétrica? Use os dados experimentais;

	\item[c)] Quais são os valores das constantes dielétricas? Use os dados experimentais;
	
	\item[d)] Para um potencial constante, a carga do capacitor aumenta ou diminui com a introdução do dielétrico? Justifique;
	
	\item[e)] Qual a finalidade do dielétrico no capacitor? Justifique.
	
\end{itemize}

\noindent{\color{red} \rule{\linewidth}{0.5mm} }\textbf{Desafio:}
Considere - Capacitor com dielétrico
\\
\textit{Será que conseguimos realizar um experimento mostrando o potencial $V$ antes e após a introdução deum material dielétrico em um capacitor didático de placas planas?}

\noindent
\textsc{Hipótestes}

Sabe-se empiricamente que a capacitância aumenta quando o capacitor é preenchido com um material dielétrico. Os primeiros a constatarem isto foram (independentemente) Faraday (1837) e Cavendish (1773). Todo dielétrico pode ser caracterizado por uma grandeza denominada \textbf{constante dielétrica}, denotada pela letra grega $\kappa$, definida por :
 $$\kappa = \frac{C}{C_0}$$
Onde $C$ e ${C_0}$ são as capacitâncias de um mesmo capacitor respectivamente com e sem dielétrico. Note que o valor mínimo $k = 1$ ocorre no caso em que o capacitor está vazio, ou seja, $ C = C_0$ O valor de $\kappa$ a temperatura de 25°C é 1,00059 para o ar, 2,25 para a parafina, 78,2 para água destilada. 
Quando um capacitor é carregado com carga $Q$ e mantido isolado, de tal forma que sua carga não pode variar, a mudança da capacitância deve ser acompanhada de uma mudança do potencial entre as placas. De fato, como $Q=C.V$ não muda, então:

$$C_0 \, V_0 = C\,V,$$

\noindent
em que $V_0$ e $V$ são os potenciais respectivamente antes e depois da introdução do dielétrico. Portanto, o novo potencial:

$$V = \frac{C_0}{C}V_0=\frac{1}{\kappa}V_0$$

\noindent
diminui por um fator $\kappa^{-1}$ em relação ao potencial $V_0$ , na ausência do dielétrico. {\color{purple}\textbf{\textit{prove isso}} }

\hskip

\noindent{\color{red} \rule{\linewidth}{0.5mm} }
\textbf{Dicas}[A]

\noindent \texttt{ O que se espera?}


\begin{figure}[H]
	\centering
	\includegraphics{img/cidepe-cxd.jpg}
	\caption{$C \times d$. Fonte: Cidepe.}
	\label{fig:imagem1}
\end{figure}

\begin{figure}[H]
	\centering
	\includegraphics{img/cidepe-cx1d.jpg}
	\caption{$C \times 1/d$. Fonte: Cidepe.}
	\label{fig:imagem1}
\end{figure}

\textbf{Dicas}[B]

A capacitância é a principal propriedade de um capacitor, e diz respeito à capacidade de armazenamento das cargas elétricas. Podemos definir Capacitância como sendo a relação entre a quantidade de cargas acumuladas e a diferença de potencial aplicada às armaduras em um capacitor. Quanto maior a capacitância, maior a quantidade de cargas elétricas que podem ser armazenadas no dispositivo.


A capacitância é medida em uma unidade denominada Farad (batizada em homenagem ao célebre físico e químico Michael Faraday), abreviada pela letra F, e no geral os capacitores utilizam submúltiplos dessa unidade, pois a capacitância de 1 F é um valor muito elevado. Um capacitor de 1F conectado a uma fonte que forneça 1V de tensão elétrica irá armazenar uma carga de 1C, que equivale a 6,24 x 1018 elétrons.

As principais unidades utilizadas para representar a capacitância de um capacitor são as seguintes:

	%tabela 5
	\begin{table}[H]
		\centering
	\begin{tabular}{l|c|c}
		\hline 
		Nome da Unidade & Símbolo & Valor equivalente em Farads\\ 
		\hline 
		Milifarad & $m F$ & $1 \times 10^{-3}\,F$ \\ 
		Microfarad & $\mu F$ & $1 \times 10^{-6}\,F$ \\  
		Nanofarad & $n F$ & $1 \times 10^{-9}\,F$ \\  
		Picofarad & $p F$ & $1 \times 10^{-12}\,F$ \\ 
		\hline 
	\end{tabular} 
		\caption{Principais unidades de Capacitância}
		\label{tab:capacitancias}
	\end{table}

Um capacitor possui capacitância de um Farad quando uma carga elétrica de um Coulomb é armazenada em suas armaduras por uma tensão elétrica de um Volt. A capacitância é sempre um valor positivo.

\bibliographystyle{plain}
  \begin{thebibliography}{1}
    \bibitem{item-1} INSTITUTO DE FÍSICA GLEB WATAGHIN. “Aula 5: Capacitância”. Disponível em
<http://midia.cmais.com.br/assets/file/original/bc19adc4984d1dd3d06412d78fe66d166e7c3514.
pdf/>. Acesso em 12 de Julho de 2018.
    \bibitem{item-2} REDAÇÃO. “Resumo de física: Capacitância e tensão elétrica”. Disponível em
<https://guiadoestudante.abril.com.br/estudo/resumo-de-fisica-capacitancia-e-tensao-
eletrica/>. Acesso em 12 de Julho de 2018.
    \bibitem{item-3} BOSONTREINAMENTOS. "Treinamentos em Ciência e Tecnologia". Disponível em <http://www.bosontreinamentos.com.br/eletronica/curso-de-eletronica/especificacoes-dos-capacitores/>. Acesso em 25 de outubro de 2020.
    \bibitem{item-4} PLATO. "Ruptura Dielétrica". <http://plato.if.usp.br/~fge0211n/Main_Site/Extras/Extras_files/Ruptura%20diele%CC%81trica.pdf>. Acesso em 25 de outubro de 2020.
  \end{thebibliography}
%\input{lei-de-ohm.tex}
%\noindent \texttt{V: 25-10-2022 17:28:00}
\section*{Experimento: - CAPACITÂNCIA}
	\section{Objetivos}
	
	Estudar o efeito capacitivo entre as placas de um capacitor variável de placas paralelas - Figura \ref{fig:cidepe-capacitor}, verificando a relação da capacitância $C=\epsilon_{0} A/d$ em função da distância $d$ entre as placas; Medir a constante de permissividade ${\epsilon}_{0}$ e medir a constante dielétrica de distintos materiais isolantes como papel, \href{https://pt.wikipedia.org/wiki/Espuma_vin\%C3\%ADlica_acetinada}{ E.V.A (\textit{Ethylene Vinyl Acetate})} e isopor.	
	\section{Material utilizado}
	
\begin{wrapfigure}{r}{0.7\textwidth}
    \centering
    \includegraphics[width=.36\textwidth]{img/cidepe-capacitor.jpg}
    \caption{Capacitor de placas paralelas com separação variável montado em isolante acrílico. \textit{Fonte: Cidepe}.}
    \label{fig:cidepe-capacitor}
\end{wrapfigure}
	
	\begin{itemize}
		\item[a)] Um capacitor variável de placas paralelas: 2,3 pF - 280 pF;
		\item[b)] Um medidor digital de capacitância;
		\item[c)] Dois fios ou cabos condutores;
		\item[d)] Um papel milimetrado;
		\item[e)] Folhas de papel ou papelão;
		\item[f)] Lâminas de isolantes E.V.A, EPS (Isopor), Vidro, Acrílico,etc;
		\item[j)] Uma régua ou trena ou paquímetro.
		
	\end{itemize}
	
\begin{comment}
\begin{figure}[H]
	\centering
	\includegraphics[scale=.3]{img/cidepe-capacitor.jpg}
	\caption{Capacitor de placas paralelas com separação variável montado em isolante acrílico. Fonte: Cidepe.}
	\label{fig:imagem1}
\end{figure}
\end{comment}


\section{Fundamentos teóricos}
	
	Se preenchermos o espaço entre as placas de um capacitor com um dielétrico (isolante), o que acontecerá com a capacitância? \textbf{Michel Faraday} (1791-1867) investigou este assunto pela primeira vez em 1837. Usando equipamento simples, ele descobriu que a capacitância aumentava por um fator $\kappa$, que ele chamou de constante dielétrica do material isolante. Outro efeito importante na introdução do dielétrico no capacitor é limitar a diferença de potencial que se pode aplicar entre as placas a um certo valor ${V}_{max}$, chamado de potencial de ruptura. Se este valor for excedido, o material dielétrico se romperá e formará um caminho condutor ({\color{red}\textbf{arco elétrico}}) entre as placas. Logo, todo material dielétrico possui uma rigidez dielétrica característica, que é o valor máximo do campo elétrico que um isolante pode tolerar sem se romper e se tornar um condutor (para o ar a rigidez dielétrica é $~3\times10^{6} \,V.m^{-1}$, ou seja tensões ou diferenças de potencial ${V}_{max}$ acima desses valores, permitem a ocorrência de um \textbf{arco elétrico} ou popularmente \textit{raio/faísca/centelha}). Na Tabela \ref{tab:rigidez-materiais} são mostradas algumas constantes dielétricas importantes.
	
	%tabela 1
	\begin{table}[H]
		\centering
	\begin{tabular}{l|c|c}
		\hline 
		Material & Constante Dielétrica, $\kappa$ & Rigidez Dielétrica $E_{max} (10^{6}\,V.m^{-1})$ \\ 
		\hline 
		Ar(1 atm) & 1,0006 & 3 \\ 
        Madeira & & 10 \\
        Borracha & & 12 \\
	Papel & 3,5 & 16 \\ 
        Poliéster & 3,6 & 21,7 \\ 
        Vidro & & 30 \\
        Mica & & 60 \\
        Teflon & & 80 \\
        
		\hline 
	\end{tabular} 
		\caption{Constante dielétrica e rigidez dielétrica}
		\label{tab:rigidez-materiais}
	\end{table}

	\section{Procedimento experimental}
	
	\subsection{A - MEDIDA DA PERMISSIVIDADE ELÉTRICA ($\epsilon_{0}$)}
	
	\begin{itemize}
		\item[a)] Certifique-se de que o capacitor esteja descarregado, fazendo contato entre as duas placas por meio de um fio ou cabo condutor;
		\item[b)] Meça o diâmetro, calcule o raio e com este a área das placas do capacitor (use: $A=\pi r^2$, onde $r$ é o \textbf{raio} das placas). Anote os resultados na Tabela \ref{tab:area-placas};
		
		%tabela 2
		\begin{table}[H]
			\centering
		\begin{tabular}{|c|c|}
			\hline 
			Diâmetro das Placas ($m$) & Área das Placas do Capacitor ($m^2$)\\ 
			\hline
			&  \\ 
			\hline 
		\end{tabular}
			\caption{\label{tab:area-placas} Diâmetro e área do capacitor}
		\end{table}
		
		\item[c)] Fazer a conexão do medidor de capacitância nas placas do capacitor. \textbf{Zerar o aparelho antes de fazer a medida};
		\item[d)] Com a chave seletora do medidor em 200 pF, estabeleça um espaçamento aproximado de 1,0 ou 10 mm entre as placas do capacitor. Anote o valor da capacitância $C_{exp}$ na Tabela \ref{tab:cte-dieletrica}; 
		\item[e)] Calcule a constante de permissividade (use a relação: $ \epsilon_{0}= C_{exp} \, d / A. \kappa_{ar} $). Anote o resultado na Tabela \ref{tab:cte-dieletrica};
		\item[f)] Calcule o erro experimental, entre o valor teórico (da literatura), e o valor experimental (medido). Anote o resultado na Tabela \ref{tab:cte-dieletrica}. 
	
		%tabela 3
		\begin{table}[H]
			\centering
		\begin{tabular}{|c|c|c|c|c|}
			\hline 
			$C_{exp}$ (pF) & $\epsilon_{0}$ (Teórico)(pF) & $\epsilon_{0}$ (experimental)(pF) & Erro (Absoluto) & Erro (\%)  \\ 
			\hline 
			& 8,85 & & & \\ 
			\hline 
		\end{tabular}
			\caption{Medição da constante dielétrica}
			\label{tab:cte-dieletrica}
		\end{table} 
	\end{itemize}
	
	\subsection{B - VARIAÇÃO DA CAPACITÂNCIA COM A SEPARAÇÃO ENTRE AS PLACAS}
	
	\begin{itemize}
		\item[a)] Certifique-se de que o capacitor esteja descarregado;
		\item[b)] Com a chave seletora do medidor em 200 pF, varie a distância entre as placas de 1 mm em 1 mm até 10 mm (5 mm em 5 mm até 50 mm no Capacitor \href{https://www.cidepe.com.br/index.php/br/produtos-interna/capacitor-variavel-de-placas-paralelas-e-cabos-0-a-255-pf-1875}{Cidepe EQ065D} ou similar). Para cada variação meça a capacitância correspondente ({\color{red}\textbf{Para uma melhor precisão, a partir da segunda medida selecione a posição da chave em $200 \,\mu F$}}). Anote os valores das capacitâncias $ C_{exp} $ na Tabela \ref{tab:c-versus-inv-d};

		%tabela 4
		\begin{table}[H]
			\centering
		\begin{tabu}{| l | X[c] | X[c] | X[c] | X[c] | X[c] | X[c] | X[c] | X[c] | X[c] | X[c] |}
			\hline 
			d ($mm$) &  &  &  &  &  &  &  &  &  &  \\ 
			\hline 
			$C_{exp} \,(pF$) &  &  &  &  &  &  &  &  &  &  \\ 
			\hline 
			1/d ($mm^{-1}$) &  &  &  &  &  &  &  &  &  &  \\ 
			\hline 
		\end{tabu} 
			\caption{Variação da capacitância em função de d e 1/d}
			\label{tab:c-versus-inv-d}
		\end{table} 
	
		\item[c)] Faça um gráfico, digital ou em papel milimetrado, de $ C_{exp} \times d $ e em seguida de $ C_{exp} \times (1/d) $;
		\item[d)] Obtenha o coeficiente angular \textbf{$\alpha$} da reta $ C_{exp} \times (1/d) $ e compare com o valor do produto $ \kappa_{ar} \epsilon_{0} A$. Anote os valores na Tabela \ref{tab:coeficientes}; 
		\item[e)] Calcule o erro experimental absoluto e percentual.
		
		%tabela 5
		\begin{table}[H]
			\centering
		\begin{tabular}{|c|c|c|c|}
			\hline 
			Coeficiente angular $\alpha$ (pF.m) & $\kappa_{ar} \epsilon_{0} A$ (Teórico)(pFm) & Erro (Absoluto) & Erro (\%) \\ 
			\hline 
			&  &  & \\ 
			\hline 
		\end{tabular}
			\caption{Coeficientes e constantes}
			\label{tab:coeficientes}
		\end{table} 
		
	\end{itemize}
	
	\subsection{C - MEDIDAS DAS CONSTANTES DIELÉTRICAS DOS ISOLANTES}
	
	\begin{itemize}
		\item[a)] Certifique-se de que o capacitor esteja descarregado;
		\item[b)] Escolher uma placa/isolante dielétrico, folha de papel. Em seguida, medir sua espessura com um paquímetro (ou um micrômetro);
		\item[c)] Meça a capacitância do ar $C_{ar}$, para uma separação miníma entre as placas, aproximadamente ~1 mm (ou ~10mm no \href{https://www.cidepe.com.br/index.php/br/produtos-interna/capacitor-variavel-de-placas-paralelas-e-cabos-0-a-255-pf-1875}{Cidepe EQ065D} ou similar) com o medidor ({\color{red}\textbf{selecione uma escala de 200 pF}});
		\item[d)] Insira o dielétrico entre as placas do capacitor, em uma posição firme, e meça a capacitância equivalente, ${C}_{d+ar}$ com o medidor ({\color{red}\textbf{selecione uma escala de $200 \,\mu F$}}). \hbox{\textbf{Obs: $C_{d}$ = capacitância com dielétrico}};
		
		\item[e)] Calcule a capacitância da placa dielétrica usando a relação: $ 1/C_{placa}=1/C_{d+ar}-1/C_{ar}$. Anote o valor na Tabela \ref{tab:cte-dieletrica-isolantes};
		\item[f)] Calcule a constante dielétrica do isolante, $ \kappa_{isolante}=C_{d}.d{/\epsilon_{0}A}$;
		\item[g)] Calcule erro experimental entre as constantes teórica (literatura) e experimental;
			\item[h)] Repita os passos a-g para os outros isolantes.
		
		%tabela 6
		\begin{table}[H]
		\centering
		\begin{tabular}{|c|c|c|c|c|c|c|c|}
			\hline 
			$d$ (mm) & $C_{ar}$ (pF) & $C_{d+ar}$ (pF) & $\kappa_{isolante}$ (teórico) &  $\kappa_{isolante}$ (exp.) & \|Erro\|  & Erro (\%) \\ 
			\hline 
			&  &  &  &  &  &\\ 
			\hline 
		\end{tabular} 
			\caption{Constante dielétrica dos isolantes}
			\label{tab:cte-dieletrica-isolantes}
		\end{table}
	\end{itemize}

\section{Exercícios}

\begin{itemize}
	\item[a)] Justifique os erros observados no experimento;
	
	\item[b)] Qual o valor da constante de permissividade elétrica? Use os dados experimentais;

	\item[c)] Quais são os valores das constantes dielétricas? Use os dados experimentais;
	
	\item[d)] Para um potencial constante, a carga do capacitor aumenta ou diminui com a introdução do dielétrico? Justifique;
	
	\item[e)] Qual a finalidade do dielétrico no capacitor? Justifique.
	
\end{itemize}

\noindent{\color{red} \rule{\linewidth}{0.5mm} }\textbf{Desafio:}
Considere - Capacitor com dielétrico
\\
\textit{Será que conseguimos realizar um experimento mostrando o potencial $V$ antes e após a introdução deum material dielétrico em um capacitor didático de placas planas?}

\noindent
\textsc{Hipótestes}

Sabe-se empiricamente que a capacitância aumenta quando o capacitor é preenchido com um material dielétrico. Os primeiros a constatarem isto foram (independentemente) Faraday (1837) e Cavendish (1773). Todo dielétrico pode ser caracterizado por uma grandeza denominada \textbf{constante dielétrica}, denotada pela letra grega $\kappa$, definida por :
 $$\kappa = \frac{C}{C_0}$$
Onde $C$ e ${C_0}$ são as capacitâncias de um mesmo capacitor respectivamente com e sem dielétrico. Note que o valor mínimo $k = 1$ ocorre no caso em que o capacitor está vazio, ou seja, $ C = C_0$ O valor de $\kappa$ a temperatura de 25°C é 1,00059 para o ar, 2,25 para a parafina, 78,2 para água destilada. 
Quando um capacitor é carregado com carga $Q$ e mantido isolado, de tal forma que sua carga não pode variar, a mudança da capacitância deve ser acompanhada de uma mudança do potencial entre as placas. De fato, como $Q=C.V$ não muda, então:

$$C_0 \, V_0 = C\,V,$$

\noindent
em que $V_0$ e $V$ são os potenciais respectivamente antes e depois da introdução do dielétrico. Portanto, o novo potencial:

$$V = \frac{C_0}{C}V_0=\frac{1}{\kappa}V_0$$

\noindent
diminui por um fator $\kappa^{-1}$ em relação ao potencial $V_0$ , na ausência do dielétrico. {\color{purple}\textbf{\textit{prove isso}} }

\hskip

\noindent{\color{red} \rule{\linewidth}{0.5mm} }
\textbf{Dicas}[A]

\noindent \texttt{ O que se espera?}


\begin{figure}[H]
	\centering
	\includegraphics{img/cidepe-cxd.jpg}
	\caption{$C \times d$. Fonte: Cidepe.}
	\label{fig:imagem1}
\end{figure}

\begin{figure}[H]
	\centering
	\includegraphics{img/cidepe-cx1d.jpg}
	\caption{$C \times 1/d$. Fonte: Cidepe.}
	\label{fig:imagem1}
\end{figure}

\textbf{Dicas}[B]

A capacitância é a principal propriedade de um capacitor, e diz respeito à capacidade de armazenamento das cargas elétricas. Podemos definir Capacitância como sendo a relação entre a quantidade de cargas acumuladas e a diferença de potencial aplicada às armaduras em um capacitor. Quanto maior a capacitância, maior a quantidade de cargas elétricas que podem ser armazenadas no dispositivo.


A capacitância é medida em uma unidade denominada Farad (batizada em homenagem ao célebre físico e químico Michael Faraday), abreviada pela letra F, e no geral os capacitores utilizam submúltiplos dessa unidade, pois a capacitância de 1 F é um valor muito elevado. Um capacitor de 1F conectado a uma fonte que forneça 1V de tensão elétrica irá armazenar uma carga de 1C, que equivale a 6,24 x 1018 elétrons.

As principais unidades utilizadas para representar a capacitância de um capacitor são as seguintes:

	%tabela 5
	\begin{table}[H]
		\centering
	\begin{tabular}{l|c|c}
		\hline 
		Nome da Unidade & Símbolo & Valor equivalente em Farads\\ 
		\hline 
		Milifarad & $m F$ & $1 \times 10^{-3}\,F$ \\ 
		Microfarad & $\mu F$ & $1 \times 10^{-6}\,F$ \\  
		Nanofarad & $n F$ & $1 \times 10^{-9}\,F$ \\  
		Picofarad & $p F$ & $1 \times 10^{-12}\,F$ \\ 
		\hline 
	\end{tabular} 
		\caption{Principais unidades de Capacitância}
		\label{tab:capacitancias}
	\end{table}

Um capacitor possui capacitância de um Farad quando uma carga elétrica de um Coulomb é armazenada em suas armaduras por uma tensão elétrica de um Volt. A capacitância é sempre um valor positivo.

\bibliographystyle{plain}
  \begin{thebibliography}{1}
    \bibitem{item-1} INSTITUTO DE FÍSICA GLEB WATAGHIN. “Aula 5: Capacitância”. Disponível em
<http://midia.cmais.com.br/assets/file/original/bc19adc4984d1dd3d06412d78fe66d166e7c3514.
pdf/>. Acesso em 12 de Julho de 2018.
    \bibitem{item-2} REDAÇÃO. “Resumo de física: Capacitância e tensão elétrica”. Disponível em
<https://guiadoestudante.abril.com.br/estudo/resumo-de-fisica-capacitancia-e-tensao-
eletrica/>. Acesso em 12 de Julho de 2018.
    \bibitem{item-3} BOSONTREINAMENTOS. "Treinamentos em Ciência e Tecnologia". Disponível em <http://www.bosontreinamentos.com.br/eletronica/curso-de-eletronica/especificacoes-dos-capacitores/>. Acesso em 25 de outubro de 2020.
    \bibitem{item-4} PLATO. "Ruptura Dielétrica". <http://plato.if.usp.br/~fge0211n/Main_Site/Extras/Extras_files/Ruptura%20diele%CC%81trica.pdf>. Acesso em 25 de outubro de 2020.
  \end{thebibliography}

\end{document}